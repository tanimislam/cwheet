\documentclass[]{article}
\usepackage[T1]{fontenc}
\usepackage{lmodern, csquotes}
\usepackage{amssymb, amsmath}
\usepackage{ifxetex, ifluatex}
\usepackage{fixltx2e} % provides \textsubscript
% use upquote if available, for straight quotes in verbatim environments
\IfFileExists{upquote.sty}{\usepackage{upquote}}{}
\ifnum 0\ifxetex 1\fi\ifluatex 1\fi=0 % if pdftex
\usepackage[utf8]{inputenc}
\else % if luatex or xelatex
\ifxetex
\usepackage{mathspec}
\usepackage{xltxtra,xunicode}
\else
\usepackage{fontspec}
\fi
\defaultfontfeatures{Mapping=tex-text,Scale=MatchLowercase}
\newcommand{\euro}{€}
\fi
% use microtype if available
\IfFileExists{microtype.sty}{\usepackage{microtype}}{}
\ifxetex
  \usepackage[setpagesize=false, % page size defined by xetex
              unicode=false, % unicode breaks when used with xetex
              xetex]{hyperref}
\else
  \usepackage[unicode=true]{hyperref}
\fi
\hypersetup{breaklinks=true,
  bookmarks=true,
  pdfauthor={},
  pdftitle={},
  colorlinks=true,
  citecolor=blue,
  urlcolor=blue,
  linkcolor=magenta,
  pdfborder={0 0 0}}
\urlstyle{same}  % don't use monospace font for urls
\setlength{\parindent}{0pt}
\setlength{\parskip}{6pt plus 2pt minus 1pt}
\setlength{\emergencystretch}{3em}  % prevent overfull lines
\setcounter{secnumdepth}{0}

\author{Tanim Islam}
\date{\today}

\begin{document}

\section{INTRODUCTION}\label{introduction}

Hello World! This is the official \texttt{README} of the
\href{https://bitbucket.org/tanim_islam/cwheet}{cwheet} (pronounced
\textit{\textbf{shweet}}) Color Wheel tool. Here I document all the
steps necessary to get started on using this tool. This is fairly
rudimentary; I have just started to look at various other
\texttt{README.md} files on \href{https://bitbucket.org}{Bitbucket},
so please be understanding!

The \textbf{cwheet} Color Wheel Tool is a fairly simple way to load
and create your own swatches of color for in CSS
format. \href{http://www.sessions.edu/color-calculator}{The Online
 Color Calculator} most closely matches the functionality of this
tool. With this tool, one can
\begin{itemize}
  \item Create, load, and save color swatches.
  \item Interactively modify colors in a color swatch by
    \begin{itemize}
      \item explicitly setting the color name and hex color definition
        in a table.
      \item dragging and clicking the selected color in an
        \href{https://en.wikipedia.org/wiki/HSL_and_HSV}{HSV color
          wheel} and color bar.        
        \item Transforming all the colors in the swatch by rotating
          through hue, increasing or decreasing saturation, or
          changing the offset color value.
    \end{itemize}
  \item Change the scale of the currently loaded, interactive HSV
    color wheel; and reset a selected color to white
    (\texttt{\#FFFFFF}). 
\end{itemize}

This document was converted from a \LaTeX source using
\href{http://pandoc.org/index.html}{\texttt{Pandoc}}, via
\blockquote{\texttt{pandoc -s README.tex -o README.md}}.

\section{INSTALLATION}
No special installation is necessary to run this file. The
requirements for this python package are located in
\texttt{requirements.txt}. To install the requirements as an user, run
the command:
\blockquote{\texttt{pip --user --upgrade -r requirements.txt}}

\section{REQUIREMENTS}
\begin{itemize}
  \item \href{http://pythonhosted.org/cssutils/}{\texttt{cssutils}}:
    Useful python utility to parse CSS files.
  \item
    \href{https://www.riverbankcomputing.com/software/pyqt/intro}{\texttt{PyQt4}}
    : Python bindings to the
    \href{http://doc.qt.io/qt-4.8/index.html}{\texttt{Qt4}}
    cross-platform toolkit.
  \item\textit{implicitly}, the \href{http://doc.qt.io/qt-4.8/index.html}{\texttt{Qt4}}
    cross-platform toolkit.
\end{itemize}

\section{TODO}
\begin{itemize}
  \item Fully developed documentation, including nice demonstration
    videos that demonstrate useful functionality.
  \item Test cases on CSS files that others send to me.
  \item The ability to have multiple color wheels within the GUI.
\end{itemize}

\section{SUPPORT}
You may contact the main developer, Tanim Islam, at
\href{mailto:tanim.islam@gmail.com}{\texttt{tanim.islam@gmail.com}}. Please
be gentle; he works on this in his free time.

\end{document}
